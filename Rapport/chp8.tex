\section{Conclusion}
\subsection{Summary}
In this report I have modelled the evolution of different classes of AGN, and calculated their emissivity of UHECRs and neutrinos
based on X-ray luminosity functions.
This emissivity was then compared to the emissivity of UHECRs and neutrinos measured at Earth.

For the distribution over luminosity one finds that Blazars and FSRQs both have a peak luminosity which is dependent on redshift around $10^{45.5} \rm erg$. One also discusses that the peak luminosity seen in Blazars is also seen in the luminosity function from a survey in the gamma ray band, which could highlight an intriguing area of additional research.
 Conversely, Bl Lacs
show an increase in numbers towards lower luminosity. Similarly, non jet-dominated AGN also show the same trend as Bl lacs, but with a break luminosity around $10^{44.5} \rm erg$ where the
slope index changes. One argues that the break luminosity in non-jet-dominated AGN is an interesting artifact illuminating a breaking point in which these sources become harder to either detect or produce, and that it warrants further investigation.


Regarding the redshift evolution one finds that  Blazars and FSRQs show a peak in their density around $z = 5$ with the brightest changing the most. In contrast, Bl Lacs 
show a negative evolution with redshift being the only AGN to do so. Non jet-dominated AGN peak around $z =0.5 $ where the peak is dependent on luminosity. The exception is Seyferts which in our chosen model 
is constant with redshift. The differences in evolution between Bl Lacs and FSRQs while being from the same class is discussed but yields no concrete answers, suggesting a new area for further research to understand these discrepancies. 

By also modelling the emissivity of our AGN one finds a change between the biggest energy injectors around redshift $z = 1$. Before this redshift the  biggest energy injectors were Blazars, while after it was surpassed by radio galaxies and Compton thin AGN. This result illuminates the fact also seen in the density evolution, that the "golden age" of Blazars is passed. 

Lastly, when looking at emissivity estimates of our AGN one finds that almost all classes of AGN can alone produce the observed diffuse flux of UHECRs and neutrinos. For UHECRs it is only FSRQs that cannot produce enough energy in X-ray to explain the observed flux. For neutrinos, it is only Bl Lacs that cannot produce enough energy in X-ray to explain the observed flux. On also tries to estimate the diffuse flux of neutrinos with a transfer model, and with this model one finds that both FSRQs and Bl lacs lie beneath the observed diffuse flux of neutrinos. One 
concludes that one needs a better correlation between X-ray luminosity, neutrinos luminosity and UHECR luminosity in order to receive a more concrete answer on the emissivity of UHECRs and neutrinos. In addition, a huge caveat in the results of emissivity is the fact that we have not included the observational properties of AGN, most notably jet orientation. The conclusion of this result is that AGN can be a source of UHECRs and neutrinos, but it does leave a lot to be desired. 



\subsection{Future outlook}
The results of this report do highlight some very interesting features in the luminosity function of X-ray selected AGN. The break luminosity in non-jet-dominated AGN can be a hint to a type of boundary in which these sources become harder to produce. 
Any further exploration of this feature would maybe need to look for the same feature in other bands, and preform a correlation analysis between the AGN selected for the different bands. What's more, the correlation between the luminosity function of Blazars in the X-ray band and the gamma ray band is also an interesting feature that could yield some interesting results. 

In regard to the emissivity estimates it became clear to the author early on that a more detailed model was needed to be made to receive a less vague answer. In still rings true that AGN can be a source of UHECRs and neutrinos, but further expansion on the transfer model which includes the orientation, and the class of AGN could create a more satisfying answer. 
It would be interesting to create an acceleration model for particles located in the X-ray corona and then propagating them through the AGN and try and capture the dynamics one might imagine happening. 
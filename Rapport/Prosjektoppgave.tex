\documentclass[11pt]{article}
%\setlength{\headheight}{13.59999pt}
\usepackage[utf8]{inputenc}

\title{Luminosity function}
\author{Henrik Andrews}
\usepackage[backend=biber,style=authoryear]{biblatex}
\addbibresource{bibfile.bib}

\usepackage{hyperref}





\usepackage{subcaption}
\usepackage[ portrait, margin=2.54cm]{geometry}
\usepackage{graphicx}
\usepackage{amsmath}
\usepackage{setspace}
\usepackage{upgreek}
\usepackage{bbold}
\usepackage{fancyhdr}
\usepackage{mathtools}
\usepackage{tabularx}
\usepackage{lipsum}
\usepackage[dvipsnames]{xcolor}
\usepackage{pdfpages}
\setlength{\parindent}{0em}
\setlength{\parskip}{1em}
\usepackage{caption}
\usepackage{multicol}	


\usepackage{float}
\newcommand{\HRule}[1]{\rule{\linewidth}{#1}}
\usepackage{listings}
\usepackage{color} %red, green, blue, yellow, cyan, magenta, black, white
\definecolor{mygreen}{RGB}{28,172,0} % color values Red, Green, Blue
\definecolor{mylilas}{RGB}{170,55,241}
\definecolor{backcolour}{rgb}{0.95,0.95,0.92}

\setstretch{1.25} %line spacing

\begin{document}

% title source: https://www.overleaf.com/project/6021d9e4632b9ef90aa8d238
\title{ \normalsize
	\HRule{0.5pt} \\
	\LARGE \textbf{{Luminosity functions of Active galactic nuclei and their emissivity of UHECRs and neutrinos}}	
	\\
	\HRule{2pt} \\ [0.5cm]		
%fix the university logo !!!!!!!!!!!!!!!!!!!!!!!!
	\vspace{6cm}
	\begin{figure}[htp]
    \centering
    \includegraphics[width=.2\textwidth]{Logo-Ntnu.svg.png}
    \end{figure}
	}

\author{
    \normalsize 
	\textbf{Henrik Døvle Andrews } \\
	Norwegian university of Science and Technology \\ 
}

\maketitle
\setcounter{page}{ 0 }

\newpage

\pagestyle{fancy}
\fancyhf{}
\setlength\headheight{14pt}
\fancyhead[L]{Henrik Døvle Andrews}
\fancyhead[R]{\leftmark}
\fancyfoot[R]{Page \thepage \:}
\setcounter{page}{1}


\maketitle

% Abstract
\begin{abstract}
%\lipsum[1] % Replace with your abstract text
Motivated by the search for the source of ultra-high-energy cosmic rays (UHECRs) and high energy neutrinos, one has in this report discussed the evolution of different classes of Active Galactic Nuclei (AGN), a potential candidate for high energy particles, and calculated their emissivity of UHECRs and neutrinos based on the X-ray luminosity. This emissivity was then compared to the emissivity of UHECRs and neutrinos as seen by ground based telescopes here on Earth. 
In addition, one has also discussed the population evolution and population distribution over luminosity of the different classes. These classes are, Blazars, Seyfert galaxies, radio galaxies, Flat spectrum radio quasars (FSRQs), BL Lacertae (Bl lac), and all Compton thin AGN. 
To do this one considered the luminosity functions of different classes generated by surveys looking in the X-ray band.
The results of the luminosity functions shows that the distribution of AGN over luminosity is different for two larger classes of AGN, jet-dominated and non-jet-dominated. Most jet-dominated AGN show a preferred luminosity around which we find the most AGN, while most non-jet-dominated AGN show an increase in numbers towards lower luminosity and a break luminosity where their slope index change. For the redshift evolution one 
also find a difference between the two classes. The total population for most jet-dominated AGN show a peak at earlier redshift than non-jet-dominated AGN. Both results have an interesting exception which are the Bl lacs who are increasing in numbers with lower redshift and also show an increase in numbers towards lower luminosities but belong to the class of jet-dominated AGN. 
The second part of the analysis in this report focuses on the emissivity UHECRs and neutrinos. One shows that assuming a similar luminosity between X-ray, UHECRs, and neutrinos, almost all classes of AGN can alone produce the observed diffuse flux of UHECRs and neutrinos. One concludes that one needs a better correlation model 
between X-ray and the particles to make a more accurate estimate of the emissivity of UHECRs and neutrinos. In addition, one wants a model that includes the observational properties of AGN, most notably jet orientation.


\end{abstract}

\newpage 
% Summary (you might use a simple section for this)
% Acknowledgments
\section*{Acknowledgments}
I would like to thank my supervisor, Professor Foteini Oikonomou, for her guidance and help throughout this project. I would also like to thank my fellow students for their help and support.
 % Replace with your acknowledgment text

\newpage
% Table of Contents (optional)
\tableofcontents

\newpage
% List of Figures
\listoffigures

% List of Tables
\listoftables


\newpage


\section{Introduction}
\subsection{Motivation and background}
Cosmic rays and neutrinos have become an important tool 
for studying the Universe. Their unique properties allow them to probe different regions better than
regular photons and are therefore a great complementary tool in observational astronomy. 
Although the information gain of detecting Cosmic rays and neutrinos are great, they are not without their challenges, 
especially when one moves into higher energies where the fluxes are low.  

The biggest challenge relating to ultra-high-energy cosmic rays (UHECR) and high energy neutrinos is their source.
Both of these particles are thought to be produced in the most energetic events/objects in the Universe, and as such the search for these 
sources are of great interest. One of the candidates is Active Galactic Nuclei (AGN). These are some of the most energetic objects in the Universe, powered by 
accretion of matter onto supermassive black holes. AGN are a fairly new object in astronomy, and one can separate them into several subclasses based on orientation, all with different potential
to produce UHECRs and neutrinos. 


\subsection{Outline}
The main goal of this report is to discuss the evolution of different classes of AGN, and calculate their emissivity of UHECRs and neutrinos based on the x-ray luminosity.
This emissivity will then be compared to the emissivity of UHECRs and neutrinos as seen by ground based 
telescopes here on Earth. To do this one will consider the distribution of the different classes in both Luminosity and redshift using luminosity functions generated by surveys looking in the x-ray band.

One starts in chapter 2 with a brief introduction to distance measures in cosmology. In chapter 3 one will discuss UHECRs and neutrinos, how they are produced, how they interact when propagating through the Universe, how they are detected and importantly the estimate of their respective emissivity.
Chapter 4 will discuss AGN, their structure, and their subclasses. In chapter 5 one introduce Luminosity functions, and the functional forms of the Luminosity functions used in this report along with their parameters.
Then, in chapter 6 one will discuss the results of the Luminosity functions. In chapter 7 one will discuss the results of the emissivity calculations, and compare them to the results of the Luminosity functions. 
Finally, in chapter 8 a conclusion and future outlook is given. 

\newpage

\input{Chapter2}

\newpage

\section{High energy particles}

In this chapter, I will discuss the different types of high-energy particles that are of interest in this paper, i.e. neutrinos and ultra-high energy cosmic rays (UHECRs).
I will briefly discuss their generation and how they are detected. Then introduce how they lose energy in their journey to the Earth, and lastly calculate the emissivity of their hypothetical sources from the ground 
observations here on earth. 

\subsection{Acceleration of high energy particles}

To reach high energy, particles need to be accelerated. 
Knowing the exact source of acceleration can be difficult if one do not know the source, but one can put constraints on any source via some simple arguments.
By arguing that the acceleration needs to be of a certain strength and that the particle being accelerated needs to stay confined within the accelerating region for long enough one can create an upper band.
This is called the Hillas criterion introduced in \cite{Hillas_1984}, and is a way of estimating the maximum energy a particle can reach in a given source.% (ref hillas)

For relativistic particles with charge $Z$ and energy $\epsilon$ in a magnetic field of strength $B$ one can define the Larmor radius


\begin{equation}
    R_L = \frac{\epsilon}{ZB}
\end{equation}

By arguing that the confinement of a particle to an accelerating region is the same as setting the Larmor radius equal to the size of the source, one can 
easily derive the maximum achievable energy for a particle as follows;% (ref M. Bustamante. https://cds.cern.ch/record/1249755/files/p533.pdf)

\begin{equation}
    \epsilon_{max} = ZBR
\end{equation}

Via this method, one can estimate the potential candidates that can produce the observed high-energy particles. 
The criterion works as an upper boundary of acceleration sources since it does not account for energy loss in the acceleration process or any type of interaction that one could expect to be in turbulent environments.
In figure \ref{fig:hillas_c} one can see the different candidates for the acceleration of two different ions, protons, and iron. One of the candidates is the AGN, which is the focus of this paper.

\begin{figure}
    \centering
    \includegraphics[width = 0.5\textwidth]{hillas_criterion.jpeg}
    \caption{Hillas criterion for proton (blue line) and iron (red line) accelerated up to $10^{20}eV$ and $10^{21}eV$ respectively. Image taken from \cite{doi:10.1146/annurev-astro-081710-102620}}
    \label{fig:hillas_c}
\end{figure}

The method of acceleration can be important for different sources, and therefore it seems useful to briefly go through some of them.

\textbf{One-shot acceleration}:
In the presence of an ordered electric field, one can continuously accelerate charged particles. This could be the feature of some astrophysical objects such as neutron stars and black holes.% (ref cern paper)


\textbf{Diffusive acceleration/ Fermi acceleration}
In regions where one has high variability in the magnetic field strength, one can accelerate particles in steps. 
This is called diffusive acceleration and the most common way of this happening is through first and second-order Fermi acceleration.
The second-order Fermi acceleration is the simplest and is based on the fact that particles can gain energy by bouncing back and forth between magnetic clouds which act as mirrors. 
This is a stochastic process and the average energy gain can be shown to be proportional $(\frac{v}{c})^2$. Here $v$ is the speed of the cloud 
and $c$ is the speed of light. This is a slow process due to the scarcity of clouds and being proportional to $v^2$, and therefore it is not a preferred method.
The first-order Fermi acceleration happens when particles collide with strong shock fronts. These shock fronts can be quite a bit faster than the interstellar clouds
and when a particle moves through the shock it gains energy proportional to $\frac{v}{c}$. In addition to this, there is a probability that the particle will stay in the accelerating region and 
experience several accelerations. 

By knowing how particles can be accelerated and their potential sources one can continue and look at the two particles in question in this paper. 
Neutrinos and UHECRs.
\subsection{UHECRs}

UHECRs are charged particles that are bombarding the Earth with energy exceeding 1 exaelectronvolt ($10^{18}$ eV) according to \cite{Alves_Batista_2019}. The origin of 
these particles is still a mystery but due to their high energies, they are thought to be extragalactic in origin.
The composition of UHECRs ranges from protons to heavier nuclei such as helium or iron, and when these particles interact with the atmosphere they produce a shower of secondary particles.
The air showers could also give extra information such as direction, but due to the nature of UHECRs, the location of their source is
difficult to pinpoint. This is because UHECRs are charged particles and therefore are deflected by the magnetic fields they encounter.

\subsubsection{Production and Energy loss}
The requirements to produce a UHECR are a charged particle and a powerful accelerator. But in order to
model them sufficiently one needs to take into account their journey to the Earth. Both during the acceleration and during the journey to Earth, the UHECRs will lose energy. 
The important parameters for this energy loss are its composition and its environment. In addition, as mentioned before, the interstellar magnetic field will also deflect the particles and therefore the direction of the particle will be changed. 
These effects are important parameters since they limit the 
distance a particle can travel before it loses too much energy, and therefore limits the local volume in which high energy cosmic rays can be produced. 
Here I will briefly discuss the different energy loss mechanisms.

\textbf{Photo-pair production}

\begin{equation}
    p + \gamma \rightarrow p + e^- + e^+
\end{equation}

For UHECRs, the most dominant sink of energy when under a certain energy threshold is the Bethe-Heitler process. In this process, a proton of sufficient energy interacts with the 
photon field in its vicinity and produces a pair of electrons and positrons. The photon field can vary from the cosmic microwave background to the generated field from different sources. 
The energy loss of this process is quite small $\sim \frac{2m_e}{m_p}= 10^{-3}$ of the original energy of the proton, but the process is very common, and therefore it is a significant energy loss over time.


\textbf{Photo-Pion production }
\begin{equation}
    p + \gamma \rightarrow \Delta^+ \rightarrow (p + \pi^0)\quad \text{or} \quad (\pi^+ + n)
    \label{eq:delta_resonance}
\end{equation}

Given enough energy the proton can interact with the photon field and produce a delta resonance. This resonance can then decay into a pion and a proton or a pion and a neutron. 
It is important since it also puts an upper limit on the UHECR energy for intergalactic particles. 
This limit, called the Greisen-Zatsepin-Kuzmin (GZK) limit comes from the UHECRs interacting with the cosmic microwave background in this delta resonance process. The limit caps proton energy at $5\times 10^{19}$ eV.
In this mechanism the original proton loses $m_p/m_\pi \approx 20\% $ of its energy resulting in a quite rapid loss of energy.

%\textbf{photodisintegration}
%mabye include this. 



%https://inspirehep.net/literature/1611251
% https://journals.aps.org/prl/abstract/10.1103/PhysRevLett.125.121106


\subsubsection{Detection}
When a cosmic ray hits the atmosphere it will interact with the air molecules and produce a cascade of particles and light that can more easily be detected than 
the original cosmic ray. In addition, since the UHECR flux at high energy is extremely low ($<$1 particle per $\rm km^2$ per year for $E > 10^{19}$) one needs a large area to collect enough data. 
The largest UHECRs detectors of present are the Pierre Auger Observatory and the Telescope Array. 

The Pierre Auger Observatory is located in Argentina and is the largest detector of its kind. It consists of 1660  Cherenkov detectors spread over 3000 km$^2$ and 27 fluorescence telescopes in four locations. With 
these instruments, the observatory is very capable of reconstructing the air showers and therefore the energy and direction of the cosmic ray. The observatory has a blind spot in the night sky 
and therefore the observatory is complemented by the Telescope Array located in Utah. The Telescope Array is a smaller observatory with 507 scintillator detectors and 3 fluorescence telescopes. Combined they have been able to map the full sky of UHECRs.  

\subsubsection{Emissivity estimates}
\label{sec:emmisivity}

Now that one reasonably understands the nature of UHECRs one can try to make tangible estimates of the UHECR sources. One such
estimate is the emissivity of UHECR sources. The emissivity is a measure of the energy released per unit time per unit volume. The question 
one can ask is what is the necessary emissivity of UHECRs to explain the observed flux here on Earth? In other words, what is the required energy injection rate per unit volume of UHECRs?


\begin{figure}
    \centering
    \includegraphics[width = 0.7\textwidth]{UHECRs.png}
    \caption{The diffuse flux of UHECRs as measured by the Pierre Auger Observatory and the Telescope Array. The flux is separated into galactic and extra galactic sources where the total spectrum follows the black dots. Image taken from \cite{Abdul_Halim_2023}}
    \label{fig:flux_UHECRs}
\end{figure}

Via observations from the Pierre Auger Observatory and the Telescope Array, one can observe and model the diffuse flux of UHECRs. The result is an isotropic flux and is represented in Figure \ref{fig:flux_UHECRs}. The functional 
form of the flux is a broken power law taken from \cite{thepierreaugercollaboration2017pierre} and is given as

\begin{equation}
    J(E_v) = \begin{cases} 
        J_0 \left(\frac{E_v}{E_{\text{ank}}}\right)^{-\gamma_1}  \text{if } E_v < E_{\text{ank}} \\
        J_0 \left(\frac{E_v}{E_{\text{ank}}}\right)^{-\gamma_2} \left(1 + \left(\frac{E_{\text{ank}}}{E_s}\right)^{\gamma_d}\right) \left(1 + \left(\frac{E_v}{E_s}\right)^{\gamma_d}\right)^{-1}   \text{if } E_v \geq E_{\text{ank}}
    \end{cases}
\end{equation}

\begin{table}
    \centering
    \begin{tabular}{|c|c|c|c|c|c|}
        \hline
        $J_0$ & $E_{\text{ank}}$ & $\gamma_1$ & $\gamma_2$ &$\gamma_d$& $E_s$\\
        \hline
        $3.3 \times 10^{-19} $ & $4.82\times 10^{18}$ & 3.14  & 4.2 & 3.14& $4.2 \times 10^{19}$  \\
        \hline
    \end{tabular}
    \caption{The model parameters for the astrophysical flux of UHECRs as measured by the Pierre Auger Observatory and the Telescope Array.}
    \label{tab:UHECR_flux}
\end{table}

with parameters in table \ref{tab:UHECR_flux}. 


By separating the 
flux into contributions from extragalactic sources and galactic sources one can estimate the required energy density in the Universe of extragalactic UHECRs. From here can define an energy loss time for a UHECR as the loss length divided by the speed of light $c$.
 The loss length is a measure of the distance a UHECR can travel before its energy drops below a certain threshold, and for our simple analysis, we will use the length of $1 Gpc$. This number is comparable in magnitude
as found by \cite{Stanev_2009} but as the loss length is dependent on initial energy and composition our number will be an approximation. 
Then the emissivity of UHECRs produced by the sources is the energy density divided by the loss time.

The previous discussion is summarized in the following equation 

\begin{equation}
    \epsilon_{\rm UHECR} = \frac{u_{\rm UHECR}}{t_{\rm loss}} = \frac{u_{\rm UHECR}}{D_{\rm loss}/c} = \frac{4\pi c \int_{E_0}^{E_{\rm max}}J_{\rm extragalactic}(E)E dE}{c D_{\rm loss}} \approx 9\times 10^{44} \frac{\rm erg}{\rm Mpc^3 \rm yr}.
\end{equation}

Here $u_{UHECR}$ is the energy density of UHECRs, $t_{loss}$ is the energy loss time, $D_{loss}$ is the loss distance, $J(E)$ is the flux of UHECRs, $E_0 = 1$ exaelectronvolt, is the minimum energy of the flux where extragalactic UHECRs become important, and $E_{max}$ is the maximum energy of extragalactic UHECRs.
The value of $\epsilon_{UHECR}$ is calculated in the script available on GitHub \cite{Andrews_2023_github} by using data from  Auger \cite{thepierreaugercollaboration2017pierre}. This emissivity is a crude estimation of the required energy injection rate of UHECRs and is meant to give a rough estimate. 
The main points of criticism are the estimate of our loss distance which does not include the composition of the UHECRs or its initial energy. Nevertheless, one receives an emissivity comparable to a more thorough analysis from \cite{PhysRevLett.125.121106} which received a value of $6*10^{44} \frac{\rm erg}{\rm Mpc^3 \rm yr}$.




\subsection{Neutrinos}

The second particle of interest is the neutrino. Neutrinos compared to UHECRs are neutral particles that are produced in various processes in the Universe.
The most common and well-known is the fusion reaction in the sun where neutrinos are produced in the pp chain. On the other hand the neutrinos of focus in this paper 
are high-energy neutrinos that are likely produced in the same sources as the UHECRs.



\subsubsection{Production and Energy loss}
The production sites of high-energy neutrinos is not clear, but they are thought to be produced in the same sources as UHECRs 
and in this section, I will go through the most probable way of producing high-energy neutrinos in sources such as AGN.

\textbf{Hadronic processes}:

Hadronic processes can release neutrinos with sufficiently high energy to explain the observations here on Earth. 
Processes such as nuclear interactions are limited by the binding energy of the nucleus and accelerating a neutrino after its production is difficult.
Therefore, a common way of producing the observed neutrinos is through the decay of pions. The most important decay is the decay of charged pions into muons and muon neutrinos as seen in equation \ref{eq:pion_decay}


\begin{equation}
    \pi^+ \rightarrow \mu^+ + \nu_\mu \rightarrow e^+ + \nu_e + \nu_\mu + \bar{\nu_\mu}
    \label{eq:pion_decay}
\end{equation}

I will discuss two possible ways of producing these pions in two different environments. 


In a proton-rich environment where the protons can accelerate up to high energies, one can produce pions through the following process
\begin{equation}
    p + p \rightarrow \begin{cases}
        \pi^+ + n+ p \\
        \pi^- + \pi^+ +p + p  \\
        \pi^0 + p+p
    \end{cases}
\end{equation}

The energy of these protons above a few GeV is enough to introduce the delta-baryon resonance, but usually one does not have a proton rich environment.
Therefore, the most efficient way of producing pions is through the already seen delta resonance when a proton interacts with a photon, this is seen in equation \ref{eq:delta_resonance}.
This process being the cooling process of UHECRs is interesting and indicates that a source that produces high energy neutrinos likely is inhabited by very energetic charged particles. 

After having produced the neutrinos it also becomes important to understand their behavior during their travel to Earth. Here I will highlight two points


\textbf{Neutrino oscillations}:
In the previous paragraph, I discussed the production of these neutrinos, but not their initial flavor.
The pion decay model is known to produce a flavor composition of $\nu_e : \nu_\mu : \nu_\tau = 1:2:0$. 
A naive thought would be an identical composition observed on Earth, but sadly this is not the case. 
The reason for this is that the neutrinos' mass state can oscillate between the different flavors. Therefore, the neutrinos produced in the source will oscillate during their travel to Earth and when they reach us one would expect a 
uniform mix of the three flavors, $ \nu_e: \nu_\mu: \nu_\tau = 1:1:1$.

\textbf{Energy loss}:
To model the travel of a neutrino of any flavor one only needs to take into account the interaction of the neutrino with the expanding universe. Since it is so weakly interacting the only 
source of energy loss the flux of neutrinos will experience is the redshift created by the expansion of the Universe. This redshift is the same as the one discussed in the previous section and the neutrinos 
behave the same way light does in this manner with a drop in energy proportional to $(1+z)$.



 

\subsubsection{Detection}
\begin{figure}
    \centering
    \includegraphics[width = 0.5\textwidth]{Ice_cube_layot.png}
    \caption{The IceCube neutrino observatory. The detector is located at the South Pole and is a large block of ice instrumented with photomultiplier tubes. Image taken from \cite{Andeen_2019}}
    \label{fig:Ice_cube}
\end{figure}

Neutrinos are weakly interacting matter particles and therefore are very difficult to detect. This makes them excellent candidates for the study of the Universe since they can travel large distances without interacting, but make them 
quite difficult to detect with high accuracy. The most famous detector and the one used in this paper is the IceCube neutrino observatory. This detector is precisely what it sounds. It is a large block of ice with a size equal to a cubic kilometer located at the South Pole.
The observatory uses the ice located deep in the South Pole as a giant Cherenkov detector. The ice is instrumented with photomultiplier tubes that can detect the Cherenkov radiation produced by neutrinos interacting with the ice. 
More precisely the observatory is fitted with 5160 photomultiplier tubes located at a depth of 1450-2450 m. The photomultipliers are divided into 86 strings of 60 modules each. The detector is also complemented by the DeepCore detector which is a denser array of photomultiplier tubes located in the center of the detector. See Figure \ref{fig:Ice_cube} for a visual representation of the detector.
The energy range for this detector is from 10 GeV to 10 EeV. The interaction of neutrinos with the water molecules in the ice can produce charged leptons (muons, electrons or taus). These charged particles if energetic enough will then produce Cherenkov radiation which can be detected by the photomultiplier tubes.


\subsubsection{Emissivity estimates}
\label{sec:emmisivity_neutrinos}

Armed with the required knowledge above one can also make simple arguments for the sources of these neutrinos based on the observed 
flux here on Earth. The flux used in this paper is the diffuse flux of neutrinos as measured by the Ice Cube observatory. The flux is shown in figure \ref{fig:flux_neutrinos}. 
For any calculations, we use the astrophysical flux as modeled as a power law. The power law is of the form 

\begin{equation}
    \Phi(E) = \Phi_0 \left(\frac{E}{E_0}\right)^{-\gamma}
\end{equation}

with $\Phi_0$ being the normalization constant, $E_0$ being the reference energy and $\gamma$ being the spectral index. The model parameters are seen in table \ref{tab:neutrino_flux} and taken from \cite{Abbasi_2022}.

\begin{table}
    \centering
    \begin{tabular}{|c|c|c|}
        \hline
        $\Phi_0$ & $E_0$ & $\gamma$ \\
        \hline
        $6.7\times 10^{-18} GeV^{-1} cm^{-2} s^{-1} sr^{-1}$ & $100 TeV$ & 2.37 \\
        \hline
    \end{tabular}
    \caption{The model parameters for the astrophysical flux of neutrinos as measured by the Ice Cube observatory.}
    \label{tab:neutrino_flux}
\end{table}

The emissivity of neutrinos is calculated in the same way as for UHECRs. The only difference is the loss time. Neutrinos do not lose energy in the same way as UHECRs and therefore the loss distance will be the size of the Universe. 
The modeled emissivity is then approximately $1.54 \times 10^{44} \rm erg/Mpc^3/yr$. 

\begin{figure}
    \centering
    \begin{subfigure}[b]{0.35\textwidth}
        \centering
        \includegraphics[width=\textwidth]{Ice_cube_flux_tot.png}
        \caption{Number of events per bin}
    \end{subfigure}%
    \begin{subfigure}[b]{0.5\textwidth}
        \centering
        \includegraphics[width=\textwidth]{Ice_cube_flux_astro.png}
        \caption{Modeled Astrophysical flux}
    \end{subfigure}
    \caption{The diffuse flux of neutrinos as measured by the Ice Cube observatory. The y-axis on the left image is the number of events per bin.  The flux is separated into contributions from atmospheric neutrinos and astrophysical neutrinos. The right image is the model astrophysical flux as measured by ICE CUBE. Images taken from \cite{Abbasi_2022} }
    \label{fig:flux_neutrinos}
\end{figure}




\newpage

\input{chp4}

\newpage 

\section{ Luminosity functions}
In this section, we will discuss the use of luminosity functions to characterize the populations of different AGNs. 
A luminosity function (LF) is a function that maps the distribution of celestial bodies, like galaxies or quasars,
based on their luminosity and corresponding comoving volume elements. These functions serve as a tool to understand the evolutionary patterns of these objects and allow us 
to predict the number density of these objects. 

%The function describes how a population varies based on luminosity but also crucially on its comoving volume element. 
Typically, the focus is on the differential luminosity function, which is defined as
\begin{equation}
    \frac{d\Psi(L,z)}{dL} = \frac{d^2N(L,V_c(z))}{dLdV_c(z)}.
\end{equation}

%The quantity of interest is now a number density which can be very useful in deriving observed flux of different objects here on earth. 
One also can change the differential of the comoving volume into a term only depending on the redshift assuming the source population is isotropic and by multiplying with the differential comoving volume element. This 
transformation goes as follows, 

\begin{equation}
    \frac{d^2N(L,V_c(z))}{dLdV_c(z)}\frac{dV_c(z)}{dz} = \frac{N(L,z)}{dLdz}.
\end{equation}


Several articles express the luminosity function in base $10$ logarithm, and we note the conversion between the two:


\begin{equation}
    \frac{d\Psi(L,z)}{dLog(L)} =  \ln (10)  Lx \frac{d\Psi(L,z)}{d(L)}.
\end{equation}


To effectively determine the LF, it's typically divided into two distinct components: a local term and a time evolution term.
 This approach involves taking the local luminosity function, calculated at a redshift 
$z=0$, and then scaling it with a function that accounts for the redshift evolution. 
The exact form of the total LF varies based on the source object, but it generally falls into two categories derived from the method of incorporating the evolution term into the local LF.
 These methods are selected based on which best represents the observed evolution.

 The two distinctions are the Pure Density Evolution (PDE) and the Pure Luminosity Evolution (PLE). 
 The PDE model modifies the local density function to reflect changes over time, 
 while the PLE model adjusts the local luminosity. The evolution is better represented by their equations and is given as 

 \begin{equation}\frac{d\Psi(L,z)}{d(L)} = 
    \begin{cases}
        \frac{d\Psi(L/e(z),z=0)}{d(L)} \quad (PLE)\\
        \frac{d\Psi(L,z=0)}{d(L)}e(z) \quad (PDE)\\
    \end{cases}
    .
\end{equation}

Here one sees the common way of representing the luminosity functions. The local luminosity function is scaled by a factor of $e(z)$ which is the evolution term.
\subsection{X-ray LF}

\begin{table}
    \centering
    \title{Parameter values for the X-ray luminosity functions}
    \begin{tabularx}{\textwidth}{|l|XXXX|XXXXX|}
        \hline

        & \multicolumn{2}{c}{\textbf{LF params}} &&&  \textbf{Evolution params} &&&&\\

        \textbf{Model} & $A$ & $L_{star}$ & $\gamma _1$ &  $\gamma _2$  & $v_1$ & $v_2$ & $z_c$ & $L_c$ & $ \alpha$\\
        \hline
        SLDDE RG & $8.375^a$ & $2.138^b$ & 2.15 & 1.10 & 4.00 & -1.50 & 1.90 & $3.981^b$ & 0.317  \\

        AMPLE-Blazar & $1.379^a$ & $1.810^b$ & -0.87 & 2.73 & 3.45 & -0.25 & & &  \\

        AMPLE-FSRQ & $0.175^a$ & $2.420^b$ & -50.00 & 2.49 & 3.67 & -0.30 & & &  \\

        APLE-BLlac & $0.830^a$& $1.000^b$ & 2.61 & &-0.79& & & &  \\
        APLE-Seyfert & $0.909^b$ & $0.61^b$ & 0.8 & 2.67& & & & &  \\
        ULDDE-CTN AGN$^c$ & $2.91^a$ & $0.93^b$ & 0.96 & 2.71& 4.78 &-1.5 &1.86 &$4.07^b$ &0.29  \\
        \hline
    \end{tabularx}
    \caption{X-ray LF parameters, $a,$ normalized by a factor of $10^{-7}$, $b,$ normalized by a factor of $10^{44}$
    c, has more factors that do not fit in the table, $z_{c2} = 3$, $\alpha_2 =-0.1$, $L_{c2} = 10^{44}$, $v_3 = -6.2 $, $\beta=0.84$ }
    \label{tab:xray_lf}
\end{table}

\begin{table}
    \centering
    \begin{tabular}{ll}
        \hline
        Model Name   & Luminosity Range (Log(L))  \\
        \hline
        SLDDE RG     & 42 - 47            \\
        AMPLE Blazar & 43 - 49          \\
        AMPLE FSRQ   & 45.5 - 49          \\
        APLE BLlac   & 44.5 - 49        \\
        APLE Seyfert & 41 - 47          \\
        ULDDE All CTN AGN & 42 - 46 \\
        \hline

    \end{tabular}
    \caption{Luminosity range for different models}

    \label{tab:lum_range}

\end{table}


%One way of calculating the neutrino flux of AGNs is based on their connecting with x-ray radiation. 
%Therefore in some literature, it is of interest to define the x-ray luminosity function for AGNs.

For a given type of celestial object, different spectral bands will be more useful than others. In the case of AGNs, 
the X-ray band is particularly significant. Therefore, several studies have focused on defining the luminosity functions of AGNs with the X-ray spectrum.

In the following, I will define the x-ray luminosity functions for various AGN classifications, including Radio Galaxies, Seyfert Galaxies, and Blazars. Furthermore, an additional breakdown will consider FSRQs and BL Lacs within Blazars.  In addition to this, 
a study by \cite{Ueda_2014} also looked at the total evolution of all Compton-thin AGNs by combining multiple surveys and research.  It will work as a reference point as well as describe the total evolution of these objects. The luminosity functions are collected from three papers \cite{Ajello_2009} and \cite{Silverman_2008}, and \cite{Ueda_2014} and their form is explained below.

\textbf{The local luminosity function}:

The local luminosity function is the luminosity function at $z=0$.
The simplest form of the local luminosity function is expressed in \cite{Ajello_2009} and is given as a power law. For our classes, it represents only the local LF for the class of BL Lacs 
and is given as

\begin{equation}
    \frac{d\Psi(L,z=0)}{dL} = \frac{A}{L_x} \left( \frac{L_x}{L_*}\right)^{1-\gamma_2}
\end{equation}

This functional form has the fewest parameters and therefore suits well for populations that have few detected sources, but has the disadvantage of not being able to capture all the details of the observed local luminosity functions when source counts increase.
For that reason a more complex local function is needed which was proposed in \cite{Ueda_2003} and is described by a double power law.
The double power law is used for the remaining classes of AGN and is given as follows


   
\begin{equation}
    \frac{d\Psi(L,z=0)}{dL} =  \frac{A}{\log(10)} \frac{1}{L_x} \left( \left( \frac{L_x}{L_*} \right)^{\gamma_1} + \left( \frac{L_x}{L_*} \right)^{\gamma_2} \right)^{-1}
\end{equation}

%The double power law introduced a break in the form of the local LF. This break is located at $L_*$ and is the luminosity where the slope of the LF changes. 


\textbf{Evolution factor}:

In addition to the local LF one also considers the evolution factor denoted $e(z)$. This factor captures the observed evolution of these objects and is the second part of the total luminosity function.

Again for the simplest evolution with the fewest parameters, a power law is used.
 $$
e(z) = (1 + z)^{v_1 }
 $$


  
Certain situations necessitate a more detailed approach to the redshift evolution. 
 As detailed in \cite{Ajello_2009}, a modified evolution is frequently employed. 
 This adaptation transforms the conventional Pure Luminosity Evolution (PLE) and Pure Density Evolution 
 (PDE) into their modified counterparts, namely Modified PLE (MPLE) and Modified PDE (MPDE).
It is within these modified frameworks that a dependence on redshift $z$ emerges in the exponent,
providing a more nuanced understanding of the evolutionary processes involved. It is given as

$$
e(z) = (1 + z)^{v_1 +v_2 z }
$$



 To expand further as described in \cite{Silverman_2008} the evolution factor of the luminosity function is not always as simple as a modified power law only dependent on the redshift $z$.
For some sources, a more complex evolution is needed. In \cite{Silverman_2008} they use a double power law to better fit the data where 
 the evolution is now not only dependent on the redshift but also on the luminosity. This then receives the apt name as a luminosity-dependent density evolution (LDDE) since it is a modified version of a (PDE)
 The functional form of the LDDE is as follows


 \begin{equation}
    e_z(z, L) = 
    \begin{cases} 
        (1 + z)^{v_1} & \text{when} z \leq z_*(L) \\
        e_z(z_*(L), L) \times \left( \frac{1 + z}{1 + z_*(L)} \right)^{v_2} & \text{when} z >  z_*(L).
    \end{cases}
 \end{equation}

 with $z(L)$ being defined as

 \begin{equation}
    z_*(L) = 
    \begin{cases} 
        z_c \left( \frac{L}{L_c} \right)^\alpha & \text{when} L \leq L_c \\
        z_c & \text{when} L > L_c .
    \end{cases}
 \end{equation}


 The expansion of the parameter space allows for easier fitting to the observed data, but comes of course with an increase in complexity and possible over fitting. 

Lastly \cite{Ueda_2014} considered an XLF for the entire population of AGNs and naturally this has a more complex evolution structure. It is also an LDDE model but with three steps instead of two which we have in \cite{Silverman_2008}.
The evolution is given as

 
\begin{equation}
    e_z(z, L) = 
    \begin{cases} 
        (1 + z)^{p_1} & \text{when} z \leq z_*(L) \\
        (1 + z_{*})^{p_1} \left( \frac{1 + z}{1 + z_*(L)} \right)^{v_2} & \text{when} z >  z_*(L)\\
        (1 + z_{*})^{p_1} (\frac{1 + z_{*2}}{1+ z_{*}})^{v_2} (\frac{1+z}{1+z_{*2}})^{v_3} & \text{when} z >  z_{*2}(L)

    \end{cases}
\end{equation}

with the exponent $p_1$ being defined as
\begin{equation}
    p_1 = v_1 + \beta(log(L)-44)
\end{equation}

with $z_{*}(L)$ being defined as

\begin{equation}
    z_*(L) = 
    \begin{cases} 
        z_c \left( \frac{L}{L_c} \right)^\alpha & \text{when} L \leq L_c \\
        z_c & \text{when} L > L_c 
    \end{cases}
\end{equation}


and $z_{*2}(L)$ being defined as

\begin{equation}
    z_{*2}(L) =
    \begin{cases}
        z_{c2} \left( \frac{L}{L_{c2}} \right)^{\alpha_2} & \text{when} L \leq L_{c2} \\
        z_{c2} & \text{when} L > L_{c2}
    \end{cases}
\end{equation}



Armed with the functional form of the total luminosity function one can now fit the parameters to the observed data. This is done in \cite{Silverman_2008}, \cite{Ajello_2009} and \cite{Ueda_2014} and their 
model name is a combination of the source paper (S, A, U), the type of model it describes (PLE, MPLE, LDDE) and the object in question. The parameters are then fitted to the data using a maximum likelihood method and the observational data of several x-ray surveys, see the cited papers for more information.
One can see the parameters for the different models in table \ref{tab:xray_lf} and the luminosity range for which the different models are valid in table \ref{tab:lum_range}. 



\newpage 

\section{Evolution}
In this section, we will be using the different luminosity functions from the previous section to calculate the evolution of the different classes of AGNs. 
By understanding these trends one can start understanding the nature of these objects, when and \colorbox{BurntOrange}{possibly how} they are created, and the amount of energy they release into the Universe. 

%By looking at the different trends for the different type of distributions one can observe the evolution and energy distribution of the different classes of AGNs. 


 


\subsection{Luminosity distribution}
 
For the different classes discussed one can integrate the differential luminosity function over redshift to retrieve the Luminosity distribution of each 
object. This distribution highlights the difference in emitting power and therefore is important for us to be able to distinguish the most powerful 
sources and their prevalence, and also any trends that might be interesting. One calculates the Luminosity density by multiplying the class-specific luminosity
function with the differential comoving volume and integrating over the relevant redshift bin. 
%that the evolution beyond the given luminosity range is not known and therefore the distribution is not complete. And deducing continued evolution
%can be done but must be taken with a grain of salt. The number of objects these functions are built upon are not very numberous and therefore
%the error bars are quite large, especially in the edges. 

The luminosity distribution is given as


\begin{equation}
    \frac{dN(L)}{dL} = \int_{z_{\text{min}}}^{z_{\text{max}}} \frac{\Psi(L, V(z))}{dL} \frac{dV(z)}{dz} dz
\end{equation}

Here, $z_{\rm min}$ and $z_{\rm max}$ are the limits of the redshift bin. By separating it into
bins of redshift one can assess how the class varies with redshift, and how a change in redshift would change the slope/trend of the distribution.  
%It is important to note 

\begin{figure}
    \centering
    \includegraphics[width = \textwidth]{new_plots/Luminosity density.png}
    \caption{Luminosity density for the  different classes of AGNs. The different classes are defined in the title as well as the chosen LF model.}
    \label{fig:LD}
\end{figure}


In figure \ref{fig:LD} one can see the luminosity density for the six different luminosity functions. The distribution is 
separated into four bins of redshift ($0<z<2,2<z<4,4<z<6,6<z<8$). %
%This is done to illuminate the evolution of the different classes at different epochs
The most interesting feature of these distributions is the difference between the break luminosity between jet-dominated classes such as Blazars and non-jet-dominated classes such as radio galaxies and Seyferts.
For the jet-dominated ones the luminosity distribution breaks at a certain point and decreases on both sides. The only exception to this is the BL Lacs which are represented as a simple power law and therefore have no
break point in their distribution. With the inclusion of more BL Lacs in future data sets it would be interesting to see if this is still the case. %The reason for the use of a single powerlaw for the BL Lacs is namly due to the lack of observations.  
The breakpoint for Blazars and FSRQs indicates some preferred 
luminosity in which one finds the most sources. This preferred luminosity also seems consistent with different epochs. To investigate this preferred energy range one could investigate different optical bands and see if the same trend is observed. 
In \cite{Narumoto_2006} they also discuss the luminosity function for Blazars but in the gamma ray band. The evolution function they use is a luminosity dependent density evolution similar to our radio galaxies and CTN AGNs. They also find a peak in the luminosity distribution, this time at a higher luminosity than the one found here.
Therefore, it could be that there is a distribution of Blazars around a certain total luminosity since one sees this trend in the x-ray and gamma ray band. 
This is not necessarily true, and one should investigate the surveys used to see the correlation between x-ray and gamma ray luminosity. \colorbox{BurntOrange}{ From the unified model of AGN one would not expect this correlation to be negative due to the different emitting regions.}
%If so it might be explained by the production of the jet and its relation to the necessary accretion rate.
%An explanation of this preferred luminosity can with some assumptions be tied to the "Blazar sequence" and idea that Blazar luminosity is strongly tied to the synchrotron peak frequency of electrons

%In \cite{Ajello_2009} where these LFs are taken from they also highlight the flattening of the luminosity distribution for the Blazar population at lower energies. This flattening is also seen here and is attributed to 
%the beaming of the x-rays from the jet.  
%An interesting point brought up in \cite{Ajello_2009} is the effect of beaming of X-rays for Blazars. This effect flattens the XLF for lower luminositie. 


For the non-jet-dominated classes, one sees an increase in numbers towards lower energies, but at some specific luminosity radio galaxies and CTN AGN introduce a break. The break luminosity is very interesting and differentiate itself 
from jet-dominated AGN since it seems to show 
a breaking point where the creation of more powerful sources becomes significantly harder and therefore less numerous. 
%The increase in x-ray luminosity is tightly connected to the amount of accretion onto the black hole and therefore a breakpoint in the energy production of similar sources is very interesting.
This breakpoint is also varying based on the redshift bin, where for higher redshift, the break is more sudden. For the Seyferts that have no evolution factor there is no distinction between lower or higher redshift.
Any change in the slope of a power-law is cause for investigation since it seems to indicate some change in possibly the structure of the sources. The softening of the break depending on redshift 
might indicate that this break is not as sudden as one would assume looking at higher redshift objects, but the change in slope index is still observed. 
The different interpretation of the luminosity distribution of these objects is fascinating and further study could illuminate why one would observe this difference, and if it is an effect of the sources, our observation of them or not an effect at all. One notes these different reasons because the author of this paper has not found any explanation to why the number of our sources suddenly drop at a certain luminosity. 
%Another interesting point of further study would be to understand the change in slope index for radio galaxies and CTN AGN. 

%The redshift bins for the Blazar population show a very interesting trend. The amount of Blazars at lower redshift seems to not change but passed the break point the total number of Blazars becomes redshift dependent, where there is a big difference between the middle epochs and the others. 
%This is also seen for the FSRQs, but the break luminosity is met with a much steeper decline. For Bl lacs, the redshift bins show a different trend. Here the number of sources is only increasing with redshift. 


\begin{figure}[H]
    \centering
    \includegraphics[width = \textwidth]{new_plots/Redshift density evolution.png}
    \caption{Density distribution for the four different classes of AGNs. The different classes are defined in the title as well as the chosen LF model.}
    \label{fig:DD}
\end{figure}



\subsection{Density distribution}

In addition to the luminosity distribution one can also calculate the number density of the different classes of AGNs. This is done by integrating the
differential luminosity function over luminosity. This will illuminate the evolution in time, or more precisely in redshift of the different classes. The integral is given as

\begin{equation}
    n(z) =\frac{N(z)}{V(z)} =  \int_{L_{\text{min}}}^{L_{\text{max}}} \frac{\Psi(L, V(z))}{dL} dL
\end{equation}

Here again, we separate into luminosity bins to see the evolution of different parts of the luminosity distribution, most notably to see the difference before and after the break luminosity for most classes. 
The results can be seen in figure \ref*{fig:DD}






%\begin{figure}
%    \centering
%    \includegraphics[width = \textwidth]{new_plots/Redshift AGN evolution.png}
%    \caption{Number evolution in terms of redshift for the four different classes of AGNs. The different classes are defined in the title as well as the chosen LF model.}
%    \label{fig:ND}
%\end{figure}


The density evolution of these objects is very interesting information due to it being closely tied to galaxy evolution. The evolution of Blazars and FSRQs differentiate themselves significantly from the other classes in figure \ref*{fig:DD}
where both FSRQs and Blazars have a positive evolution with a peak in density around redshift $z=5$. The total counterpart to this is the third class of jet-dominated AGN, the Bl Lacs. Here one sees a negative evolution with 
an increase in density in the more recent epochs. In a paper by \cite{Garofalo_2019} he talks about the different evolutionary paths that generate the respective classes of Bl lacs and FSRQs which could be the origin of this discrepancy. 
While both classes belong to the parent class of Blazars the evolutionary path of FSRQs is thought to come from FRII radio galaxies while the evolutionary path of BL Lacs is from FRI radio galaxies. As mentioned in section \ref*{sec:jets} the difference between FRI and FRII jets is thought to be the accretion efficiency. 
In addition to this a paper by \cite{Wei-Hao_2003} which studied the correlation between accretion rates and bolometric luminosity mentions that the nature of the central black hole and its rotation will have an effect on the accretion rate. Drawing from this one could indicate that the difference in evolution stems from the different evolutionary paths of their central engine, Kerr black hole or not.  
This is for the time being not an accepted explanation and still a topic of debate. 
Another try to explain how accretion efficiency is related to the central engine is in \cite{Raimundo_2012} where they discuss a finding that showed the efficiency of the accretion being proportional to the mass of the black hole.  
More precisely they found this relation: $\eta \propto M^{0.5}$. Although this could help explain our density evolution of FSRQs and Bl Lacs by allowing FSRQs to host bigger black holes they also mentioned that this effect might be an artifact of the parameter space used. 
On the other hand one could also try to look at the evolution of material that can be accreted around the central black hole to possibly start unraveling the different evolutionary paths.
From this it is only reasonable to conclude that this difference in evolution is captivating and is prone to an interesting answer. 


For the Blazar population, one notices the same trend as for the luminosity distribution. The luminosity bin before the break luminosity stays more constant than the ones after the break. The reason for such an evolution 
would likely be tied to the same mechanism driving Bl lacs and FSRQs. Due to the decline of FSRQs, one should also expect the
higher-end luminosity of Blazars to follow.

For the non-jet-dominated AGNs, one finds a different story. Here the redshift peak, if any, is at around $z=0.3$ where the peak is dependent on the luminosity bin. Lower luminosity AGN peaks at lower redshift. Therefore, the trend of density 
seems to be going toward lower-power radio galaxies and Compton-thin AGNs. 
What is very interesting is comparing this evolution to 
the evolution of star formation. From \cite{Madau_2014} the star formation rate peaks at around 3.5 billion years after the Big Bang, 
or around redshift $z= 1.9$. This is in stark contrast to our sources where only the most luminous radio galaxies and Compton thin AGNs peak at this redshift. 
The star formation rate then places itself in between the two peaks between jet-dominated and non-jet-dominated AGNs which opens up for interpretation. 
%This is very interesting since then the presence of efficient accreting AGN such as our FSRQs would have been more numerous before the peak of star formation. 
%And on the opposite side our CTN AGNs peak later.  


\subsection{Expected luminosity}
\label{sec:Expt_lum}
From the luminosity function, one can also calculate the expected luminosity of a source class at different redshifts. This is important since it will directly relate to the 
power injection of the different epochs and from this one can calculate an expected emissivity of the different classes of AGNs. 
The expected luminosity of each group can be calculated with the following formula

\begin{equation}
    \langle L \rangle = \frac{\int_{L_{\text{min}}}^{L_{\text{max}}} L \frac{\Psi(L, V(z))}{dL} \frac{dV(z)}{dz} dL}{\int_{L_{\text{min}}}^{L_{\text{max}}} \frac{\Psi(L, V(z))}{dL} \frac{dV(z)}{dz} dL}
\end{equation}

furthermore, the emissivity is given as


\begin{equation}
     \epsilon  = \int_{L_{\text{min}}}^{L_{\text{max}}} L \frac{\Psi(L, V(z))}{dL} \frac{dV(z)}{dz} dL
\end{equation}

The different luminosity ranges are the same as before and are given in table \ref{tab:lum_range}. The results are shown in figure \ref{fig:EL}.

\begin{figure}
    \centering
    \includegraphics[width = 0.8\textwidth]{new_plots/Luminosity and Emissivity.png}
    \caption{Expected luminosity and emissivity for the four different classes of AGNs. The different classes are defined in the title as well as the chosen LF model.}
    \label{fig:EL}
\end{figure}



The expected luminosity is shown at the top in figure \ref*{fig:EL} where it shows the expected power output of the different classes. Here one sees that FSRQs are indeed the most luminous AGN and 
that they represent some of the most luminous objects in the Universe.
The expected luminosity also shows the evolutionary trend of Blazars where they are now tending towards lower average luminosity. 
All classes remain fairly constant, but radio galaxies and CTN AGNs both have a decline in expected luminosity after the star formation peak at $z=1.9$. This could be inferred from figure \ref*{fig:DD} but is more clearly seen here.
%The constancy of most objects across redshifts is comforting since it means that one could use a general model for any of the objects at any redshift without needing to account for the redshift. 
%This is not the case for the Blazars but since they are a combination of different objects any general model would be hard to find. On the other hand, any model for FSRQs and Bl lacs would not need to account for redshift and the varying parameters that come with that. 
%An important point to remeber is that this luminoisty is directly tied to the x-ray production of these sources. Therefore the expected luminoisty also shows us that 
%jet-dominated AGN produce more x-rays than non-jet-dominated AGN. This is not suprising but it is a indcication that these x-rays are not only produced in the hot corona, but also in the jet. 
%While for all compton thin AGNs one would expect the x-ray luminosity to be dominated by the hot corona.

The emissivity of each class as a function of redshift is shown in figure \ref*{fig:EL} at the bottom. Here it shows a change in dominance between Blazars and our CTN and radio galaxies.
%The change in dominance happens around redshift $z=1$ and is a result of the different evolution of the different classes.
This change that happens around redshift $z=1$ is a result of the different evolutionary paths of our sources seen in figure \ref*{fig:DD} and should affect the diffuse astrophysical flux of UHECRs and neutrinos. 
Due to the different orientations of the sources one should expect them to create UHECRs and neutrinos differently. From this any decline in emissivity of for example Blazars would separate the expected flux of neutrinos and UHECRs if they are produces in the same source. This is due to the energy loss mechanisms 
discussed in section \ref{sec:emmisivity}. In order to test this one would need a way of correlating the production of neutrinos with UHECRs, and this requires to model the different ways of not only accelerating particles but also the direction of acceleration. 
Two possible ways of accelerating particles which compliments the different classes discussed in this report is the Blandford-Znajek process which produces jets, and outflow mechanics which can in theory take any direction but are usually discussed as outflows into the plane of the host galaxy.
For jet-dominated AGNs the most accepted theory of energy extraction from rotating black holes is discussed in \cite{Blandford_1977}. This method of acceleration which is produced by ordered electric fields drives the accelerated particles into jets. One cares about this since any emission of particles especially neutrinos would therefore be highly pointed and one would only expect 
AGNs with a jet pointed in our line of site to produce neutrinos we would detect. 
The method of outflows discussed in \cite{Laha_2021} is another way of accelerating particles \colorbox{BurntOrange}{and is akin of reconnection events that happen in the sun}. The problem one faces here is 
therefore the uncertainty of accelerating our particles enough. The mechanism of outflow is not certain enough to produce the highest energy UHECRs and neutrinos. The orientation of these outflows can be also be difficult to determine therefore it becomes harder to use these outflows as consistent methods of generating our desired particles.
These different ways of accelerating our particles are important to consider later when talking about the diffuse flux of UHECRs and neutrinos since they would be quite dependent on the class of AGN.


 \colorbox{BurntOrange}{A note for Foteini: The last paragraph is in my opinion important, but I find it awkwardly placed, and} - \\
 \colorbox{BurntOrange}{I feel like I do not know enough to keep it in, please correct me if I have written anything wrong or weird.}
%The emissivity is a good indicator of what objects might dominate the different epochs, and it is clear that at higher redshift $z>2$ our jet-dominated AGNs are the most powerful emitters of x-rays. 
%An interesting point is how this increase in emissivity of different classes of AGNs would affect the outflow mechanism of the AGNs and how this would affect the surrounding galaxy. In \cite{Laha_2021}
%they talk about the open questions regarding the outflow mechanism of AGNs and their feedback mechanism with their host galaxy. To add to these open questions one could ask how the change in x-ray emissivity in the different classes of AGNs would affect the same mechanisms.



\newpage


\section{Exotic particle creation}
\subsection{UHECRs emissivity}
With the calculated emissivity for the different groups, there is now the possibility to look from an energy budget viewpoint into the possibility of AGNs being the origin of UHECRs. The 
reasoning is quite simple but for the AGNs to be the origin of UHECRs they must be able to produce the necessary emissivity. 

From the calculation in \ref{sec:emmisivity} the emissivity of UHECRs is given as $1.73 \cdot 10^{44}\frac{\rm erg}{\rm Mpc^3yr \rm s }$ this was calculated from the observed flux of UHECRs from the Pierre Auger observatory \cite{thepierreaugercollaboration2017pierre}.
In this calculation, one needed to confine the area in which these sources could be produced to take into account the energy losses these particles experience. The same argument must be used 
for our emitting sources and therefore one must use the emissivity of our sources at a redshift very close to Earth. To get a comparable emissivity one evaluates therefore the emissivity at redshift $z=0.01$. The result is shown in figure \ref*{fig:flux_UHECRs}

\begin{figure}[H]
    \centering
    \includegraphics[width = 0.7\textwidth]{new_plots/L_n_uhecr_calc.png}
    \caption{UHECR emissivity for the four different classes of AGNs.}
    \label{fig:UHECR}
\end{figure}

This figure shows that most classes of AGN produce a total emissivity in X-ray comparable to the one energy detected by the Pierre Auger observatory. The only exception is the FSRQs which are not numerous enough at this redshift. 
To criticize this very crude estimate, one must first note that the correlation between X-ray luminosity and UHECR luminosity is not well-defined and should include parameters that are not accounted for.
In addition to this one has not done any separation between jet-dominated and emission-line AGNs, and even though our emission-line AGNs are capable of producing the required x-ray luminosity
the mechanism of transferring this energy into UHECRs is not well understood. A very interesting point is that the only candidate not able to produce the required emissivity, the FSRQs, are the ones where according to \cite{Wei-Hao_2003} one could find a Kerr black hole. 
A Kerr black hole is needed in mechanisms such as the Blandford-Znajek mechanics which is a possible way of accelerating and beaming particles into jets. One notes that it is not only that mechanism that could accelerate UHECRs and 
the outflows discussed in \cite{Laha_2021} would have an acceleration mechanism as well. The author of this paper notes that the energy gain in these outflows is not known and therefore one cannot say if they are powerful enough to produce extra galactic UHECRs. 

Nevertheless, the result does not rule out the possibility of AGNs being the origin of UHECRs. 


\subsection{Neutrino emissivity}
Similarly, for the UHECRs, we calculated the local emissivity for the neutrinos in section \ref{sec:emmisivity}. The result was $1.2 \cdot 10^{44}\frac{erg}{Mpc^3yr}$ which is a factor similar to that of UHECRs.
In the calculation, the diffuse neutrino flux on Earth was taken from the IceCube observatory \cite{Abbasi_2022} and the energy range was taken to be $1TeV - 10PeV$ corresponding to the astrophysical neutrino flux.
The difference between the UHECR flux to the neutrino flux is the energy loss mechanism. The effect of a very limited energy loss mechanism means that the emitting area is now the whole universe. To reach a comparable emissivity one must therefore take a redshift-dependent average of the sources over the whole universe.
One does this by scaling the emissivity at redshift $z$ with the corresponding energy loss for a neutrino from that redshift given as $(1+z)$ and then taking the average emissivity to get a comparable emissivity.
The resulting figure is shown in figure \ref{fig:neutrino}.

\begin{figure}[H]
    \centering
    \includegraphics[width = 0.7\textwidth]{new_plots/L_n_neut_calc.png}
    \caption{Neutrino emissivity for the four different classes of AGNs.}
    \label{fig:neutrino}
\end{figure}

This figure shows that the neutrino flux can be produced by all classes except 
the BL Lacs. This is an effect of the averaging since the BL lacs have a negative evolution.
The opposite is the FSRQs which now can produce the required emissivity.

In addition to the average picture given in figure \ref*{fig:neutrino} one can also calculate the diffuse neutrino flux from the different classes of AGNs. This is done by modifying the transfer function defined in \cite{Palladino_2020} and this is given as


\begin{equation}
    \frac{d\phi_\nu}{dE_\nu} = \int_0^{z_{max}} \frac{D_H}{E(z)} \frac{L(E_\nu (1+z),<L_x>(z))}{(1+z)^2} \rho(z) dz
\end{equation}

Here $D_H$ is the Hubble distance, $E(z)$ is the function defined in section \ref{sec:comoving_distance}, $L(E_\nu (1+z), <L_x>(z))$ is a power law representing the neutrino flux at the source, which when integrated reproduces the average source luminosity at redshift $z$, and $\rho(z)$ is the number density of the sources at redshift $z$.
With this function, we can calculate the expected diffuse flux of neutrinos from the sources. The difference between \cite{Palladino_2020} and I, is the inclusion of a luminosity dependence in the power-law function. This is done to account for the different average luminosity of the different classes of AGNs which was assumed constant in \cite{Palladino_2020}. 
The assumption of a constant luminosity is not a bad one since the luminosity of the different classes is not changing significantly over the redshift range, but it is still a simplification, especially for the Blazar class. The form of the spectra is taken to be identical to the observed neutrino flux by ICE CUBE which is defined in section \ref*{sec:emmisivity_neutrinos}.
The result is shown in figure \ref{fig:neutrino_diffuse}.
\begin{figure}[H]
    \centering
    \includegraphics[width = 0.7\textwidth]{new_plots/diffuse_fluxes_neutrino_no_cutoff.png}
    \caption{Diffuse neutrino flux for the four different classes of AGNs.}
    \label{fig:neutrino_diffuse}
\end{figure}

What one sees here is almost the same result as the crude average which is good. The FSRQs and Bl lacs are now not able to produce the diffuse flux but the rest are. 
The fact that all sources are overshooting the observed neutrino flux would be a problem if we had a more constrained solution. Any model that overshoots could not be the source since it is not what we observe, but in our case, 
our solution rests on the fact that the neutrino luminosity is equal to the x-ray, an assumption easily broken and without nuance. Therefore, the only concrete conclusion one can draw from this is that the neutrino flux can be produced by the AGNs since they can produce the required emissivity.
Several papers such as \cite{Kurahashi_2022} talk about the lack of any anisotropy in the observed diffuse neutrino flux. Such an anisotropy would introduce a required density of sources. This would be a problem for the more obscure sources such as FSRQs and could further limit our predictions.

This result in figure \ref*{fig:flux_neutrinos} is obtained differently than in figure \ref*{fig:neutrino} and therefore the agreement between them is a good sign. The argument that the x-ray luminosity should have the same value as the neutrino luminosity is not a bad first guess, but it does leave a lot to be desired. 
This flux model does not incorporate any parameters of the AGN other than emitting strength, and therefore it is not a very nuanced model. A first fix could be a model that includes the jet orientation and most importantly the acceleration mechanism one could imagine happening. What this 
crude model can show is that these objects do produce enough power. To make the estimation better without doing much more work, one could look at the gamma-ray luminosity functions of the same sources and see if they can produce the required neutrino flux. Gamma-ray production is a natural consequence if one is using the pion decay model with the proton delta resonance
for neutrino production. In this way, one could more easily constrain the neutrino production with the gamma-ray production. 
This is however outside the scope of this paper and will be left for future work. 

\subsubsection{AGN as the origin of the astrophysical particles}
Both the diffuse flux of neutrinos and UHECRs estimate an emissivity that is comparable to the x-ray emissivity of our sources. The correlation between x-ray luminosity and the production of these ultra-high energy 
particles is also not unfounded since both require a high number of charged particles. For these arguments, one could imagine that AGNs of all classes can produce these particles. 



\newpage

\section{Conclusion}
\subsection{Summary}
In this report I have modelled the evolution of different classes of AGN, and calculated their emissivity of UHECRs and neutrinos
based on X-ray luminosity functions.
This emissivity was then compared to the emissivity of UHECRs and neutrinos measured at Earth.

For the distribution over luminosity one finds that Blazars and FSRQs both have a peak luminosity which is dependent on redshift around $10^{45.5} \rm erg$. One also discusses that the peak luminosity seen in Blazars is also seen in the luminosity function from a survey in the gamma ray band, which could highlight an intriguing area of additional research.
 Conversely, Bl Lacs
show an increase in numbers towards lower luminosity. Similarly, non jet-dominated AGN also show the same trend as Bl lacs, but with a break luminosity around $10^{44.5} \rm erg$ where the
slope index changes. One argues that the break luminosity in non-jet-dominated AGN is an interesting artifact illuminating a breaking point in which these sources become harder to either detect or produce, and that it warrants further investigation.


Regarding the redshift evolution one finds that  Blazars and FSRQs show a peak in their density around $z = 5$ with the brightest changing the most. In contrast, Bl Lacs 
show a negative evolution with redshift being the only AGN to do so. Non jet-dominated AGN peak around $z =0.5 $ where the peak is dependent on luminosity. The exception is Seyferts which in our chosen model 
is constant with redshift. The differences in evolution between Bl Lacs and FSRQs while being from the same class is discussed but yields no concrete answers, suggesting a new area for further research to understand these discrepancies. 

By also modelling the emissivity of our AGN one finds a change between the biggest energy injectors around redshift $z = 1$. Before this redshift the  biggest energy injectors were Blazars, while after it was surpassed by radio galaxies and Compton thin AGN. This result illuminates the fact also seen in the density evolution, that the "golden age" of Blazars is passed. 

Lastly, when looking at emissivity estimates of our AGN one finds that almost all classes of AGN can alone produce the observed diffuse flux of UHECRs and neutrinos. For UHECRs it is only FSRQs that cannot produce enough energy in X-ray to explain the observed flux. For neutrinos, it is only Bl Lacs that cannot produce enough energy in X-ray to explain the observed flux. On also tries to estimate the diffuse flux of neutrinos with a transfer model, and with this model one finds that both FSRQs and Bl lacs lie beneath the observed diffuse flux of neutrinos. One 
concludes that one needs a better correlation between X-ray luminosity, neutrinos luminosity and UHECR luminosity in order to receive a more concrete answer on the emissivity of UHECRs and neutrinos. In addition, a huge caveat in the results of emissivity is the fact that we have not included the observational properties of AGN, most notably jet orientation. The conclusion of this result is that AGN can be a source of UHECRs and neutrinos, but it does leave a lot to be desired. 



\subsection{Future outlook}
The results of this report do highlight some very interesting features in the luminosity function of X-ray selected AGN. The break luminosity in non-jet-dominated AGN can be a hint to a type of boundary in which these sources become harder to produce. 
Any further exploration of this feature would maybe need to look for the same feature in other bands, and preform a correlation analysis between the AGN selected for the different bands. What's more, the correlation between the luminosity function of Blazars in the X-ray band and the gamma ray band is also an interesting feature that could yield some interesting results. 

In regard to the emissivity estimates it became clear to the author early on that a more detailed model was needed to be made to receive a less vague answer. In still rings true that AGN can be a source of UHECRs and neutrinos, but further expansion on the transfer model which includes the orientation, and the class of AGN could create a more satisfying answer. 
It would be interesting to create an acceleration model for particles located in the X-ray corona and then propagating them through the AGN and try and capture the dynamics one might imagine happening. 

\newpage

\printbibliography
\end{document}

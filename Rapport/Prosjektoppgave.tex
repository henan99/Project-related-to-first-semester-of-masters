\documentclass[11pt]{article}
\setlength{\headheight}{13.59999pt}
\usepackage[utf8]{inputenc}

\title{Luminosity function}
\author{Henrik Andrews}

\usepackage{natbib}


\usepackage[ portrait, margin=2cm]{geometry}
\usepackage{graphicx}
\usepackage{amsmath}
\usepackage{upgreek}
\usepackage{bbold}
\usepackage{fancyhdr}
\usepackage{mathtools}
\usepackage{tabularx}
\usepackage{pdfpages}
\setlength{\parindent}{0em}
\setlength{\parskip}{1em}
\usepackage{caption}
\usepackage{multicol}

\usepackage{float}
\newcommand{\HRule}[1]{\rule{\linewidth}{#1}}
\usepackage{listings}
\usepackage{color} %red, green, blue, yellow, cyan, magenta, black, white
\definecolor{mygreen}{RGB}{28,172,0} % color values Red, Green, Blue
\definecolor{mylilas}{RGB}{170,55,241}
\definecolor{backcolour}{rgb}{0.95,0.95,0.92}
\begin{document}

% title source: https://www.overleaf.com/project/6021d9e4632b9ef90aa8d238
\title{ \normalsize \textsc{AREA}
	\\ [2.0cm]
	\HRule{0.5pt} \\
	\LARGE \textbf{\uppercase{Theme}}		
	\\ Title\\
	\HRule{2pt} \\ [0.5cm]		
%fix the university logo !!!!!!!!!!!!!!!!!!!!!!!!
	\vspace{6cm}
	\begin{figure}[htp]
    \centering
    \includegraphics[width=.2\textwidth]{Logo-Ntnu.svg.png}
    \end{figure}
	}

\author{
    \normalsize 
	\textbf{Henrik Andrews } \\
	Sorbonne Université \\ 
}

\maketitle
\setcounter{page}{ 0 }

\newpage

\pagestyle{fancy}
\fancyhf{}
\setlength\headheight{12pt}
\fancyhead[L]{Henrik Andrews}
\fancyhead[R]{Luminosity functions}
\fancyfoot[R]{Page \thepage \:}
\setcounter{page}{1}
\bibliographystyle{apalike}

\maketitle
-abstract 
-sammendrag
-acknowledgments
-list of figure
-list of tables


\section{Introduction}
\bibliography{bib}
\section{Comsological distances}
In order to investigate sources very far away from an observer it is important to understand the influence this distance has on your desired observable. Therefore in astrophysics and astronomy in general there are defined terms created to take into account the effects of f.ex the ever-expanding universe. 


\subsection{comsographic parameters}

The most notorious parameter is the Hubble constant $H_0$. This parameter set the recession speed of a point at proper distance d and our current position via this equation.
\begin{equation}
    v = H_0 d 
\end{equation}
The subscript $0$ refers to the present epoch since in general $H_0$ changes with time. The value of $H_0$ is quite debated so I will  follow the nomenclature of D. Hogg and write it as a parameterized equation. 
$$
H_0= 100\frac{km}{s}\frac{1}{Mpc} h 
$$
where $h$ is a dimensionless number that according to current knowledge  can take the value between $0.5$ to $0.8$  

The Hubble constant has units of inverse time and therefore one defines the Hubble time as 
$$
t_H = \frac{1}{H_0}
$$
and also has units of speed, and therefore we can define the Hubble distance. 

$$
D_H = \frac{c}{H_0}
$$

\subsection{Components of the universe}
In this paper and in most articles one refers to the flat lambda CDM model to parameterize the contents of the universe and by extension the properties of its expansion. Here two important parameters to define are the mass density of the universe $\rho_0$ and the cosmological constant $\Lambda$. These variables which change with time also define the metric tensor in general relativity and allow us to model the curvature of the universe given an initial configuration.  njaaa, skriv på nytt. 

one can write these into dimensionless variables as such

$$
\Omega_m = \frac{8\pi G\rho_0}{3H_0^2}
$$

$$
\Omega_\Lambda = \frac{\Lambda c^2}{3H_0^2}
$$



In general one has a third density parameter $\Omega_k$ which defines the curvature of spacetime and is defined from 

$$
\Omega_m + \Omega_\Lambda + \Omega_k = 1
$$



The flat lambda CDM has three components, the dark energy density, the dark matter density, and the ordinary matter parameter. In the rest of this paper and other articles used by this dissertation the values of these components are represented by their $\Omega$ parameter and one has a universe with $\Omega_\Lambda = 0.7$, and $\Omega_m = 0.3$ also known as the flat lambda since the curvature parameter is $0$


\subsection{Redshift}
Redshift is defined as the fractional Doppler shift of emitting light. The Doppler effect is a known effect on different observables in our universe where the relative motion of sources to observers will impact the observable. The redshift is quantified for a light source as 

\begin{equation}
    z = \frac{\nu_e}{\nu_o}-1 = \frac{\lambda_o}{\lambda_e}-1
\end{equation}

Here $o$ refers to the observed quantity and $e$ the emmited. If one want to connect this redshift to the velocity of the observed object one needs to go to general relativity. There is an analog in special but one omits it here since one does not use it. The important factor is that redshift is an observable and can help us determine distances of objects. especially if they are far away the relative velocity becomes negligible and only the effect of the expansion of the universe is important. 


\subsection{Comoving distance}
The comoving distance or more clearly the line of sight distance for an observer locted here
 at earth is a foundational distance measure in cosmography. All other distance measures can be 
 derived from it. One derives it by defining the small co-moving distance $\delta D_c$. 
 This quantity defines the distance between two objects that remains constant when both objects expand with the Hubble flow.
  One can think of it as a proper distance from relativity since it is constant in all "time-frames". 
  If one wants the total comoving distance one integrates all $\delta D_c$ in the line of sight from $z= 0$ to the object. 
From  \cite{hogg2000distance} one defines the function 

\begin{equation}
    E(z)  = \sqrt{\Omega_m(z+1)^3 +\Omega_k (1+z)^2 + \Omega_\Lambda  }
\end{equation}


This function is defined by the density parameters defined above and also the redshift $z$. One can also relate this to the measured Hubble constant as measured by an hypothetical observer at redshift $z$ via $ H(z) = H_0 E(z)$. 


One then recives the comoving distance $D_c$ from 
\begin{equation}
    D_c =D_H \int_0^z\frac{dz}{E(z)}
\end{equation}
\subsection{Comoving distance part 2}
$D_c$ is the line of sight of an object and its observer but given different space-time geometry that line is distorted. If one looks a two objects, both at redshift z the distance between them will be given as as a function of the angle between them. If they are sperated by an angle $d\theta$ then the distance between them will be $d\theta D_m$ where $D_m$ is the comoving distance

%Therefore for different curvature, one need to map the geodesic onto our simplistic view. Therefore one gets the three different space geometry cases

$$
D_m =
\begin{cases}
  D_h\frac{1}{\sqrt{\Omega_k}}sinh(\frac{\sqrt{\Omega_k}D_c}{D_H}) & \text{if } \Omega_k > 0 \\
  D_c& \text{if } \Omega_k = 0 \\
  D_h\frac{1}{\sqrt{|\Omega_k|}}sin(\frac{\sqrt{|\Omega_k|}D_c}{D_H}) & \text{if } \Omega_k < 0
\end{cases}
$$

The different cases are dependent on the curvature of the universe, and one can see that one enters hyperbolic geometry or spherical geometry based on the different curvatures. The true curvature of the universe is still a mystery but recent studies suggest that it is flat, as expected. 


\subsection{Angular distance}


THe angul


\subsection{Luminosity distance}
The luminosity distance $D_l$ is defined through the relation between 
the bolometrix flux $F$ of a source and its bolometric luminosity $L$. 
The flux is the amount of energy per unit time per unit area and the luminosity is the total amount of energy per unit time. 
The luminosity distance is defined as
\begin{equation}
    D_l = \sqrt{\frac{L}{4\pi F}}
\end{equation}

In more clearer words it describes the total loss in energy due to lights travel across an expanding universe. 
The total observed flux an observer will see will be different from different distances.  the intrisic luminosity emmited at the source


It is related to the transverse comoving distance via 

\begin{equation}
    D_l = (1+z)D_m
\end{equation}

This of course is for bolometric quantities, but if one wants to calculate the spectral 
flux/ differential flux one need to take into account a correction. This correction comes 
from the fact that one is viewing a redshifted object. The object is emmiting in a different band than 
observed. The spectrum of the differnetial flux $F_\nu$ is related to the spectral luminosity via
\begin{equation}
    F_\nu = (1+z) \frac{L_{(1+z)\nu}}{L_\nu}\frac{L_\nu}{4\pi D_l^2}
\end{equation}

If one wishes to translate an observed spectral flux of particles to a density distriubtion one needs to take into account this redshift effect


\section{Particle theory}

\section{Neutrino production}

We will be discussing two types of neutrino generation. The pp chain and p$\gamma$ production
\section{Types of AGNs}

\subsection{Active galactic nuclei}



Active Galactic Nuclei (AGNs) is an interesting field in astrophysical studies. 
Since their discovery, there has been rapid advancement in understanding these phenomena.
Today, AGNs are known to be among the brightest entities in the night sky,
but they only gained significant attention in the 1950s. 
This shift occurred with the arrival of new radio observations, which revealed a new type of quasi stellar
object through the discovery of Quasars.

Initially, these luminous objects, characterized by broad, 
unidentifiable spectral lines, were enigmatic to scientists in the early 1960s. 
However, with the identification of more sources and their optical parts, 
it became clear that these were not stars but a distinct class of celestial objects. 
Research done by M. Schmidt on of the emission lines from 
the Quasar 3C 273 opened the interpretation of these celestial objects. 
He found that the emmision lines of quasars were similar to hydrogen, but were redshifted by a factor of 0.158,
an exceptionally high value at the time. Observations at the same time also revealed significant 
variability in quasar luminosity, suggesting that these objects were no larger than one light year across. 

These observations lead to the speculation of super luminous objects located very far away from earth. The problem was that such objects
had no reasonable explaination at the time. It was not until the late 1960 early 1970s when modern cosmology was 
afoot that more of these issues were resolved.

Observation of the surrounding galaxy of AGNs with matching redshift and observation of gravitational lensing cemented 
the distances of these objects. In addition the modern view of black holes which had only been a theory in the 1950s came to
furition and the modern model of a AGN was born. This modern perspective views AGNs as supermassive black holes that
accrete matter from surrounding accretion disks. This accretion releases large amounts of energy and has also according to 
processes such as  the Blandford-Znajek process, been shown to produce relativistic jets, when the black hole is rotating.

In the most recent times a landmark achievement was achieved in March 2021, when scientists associated with the Event Horizon Telescope project 
presented the first image of the supermassive black hole at the center of the Messier 87 galaxy, located 55 million light-years away.
This image, showing a bright ring surrounding a dark central region, aligns with predictions for an accreting supermassive black hole, 
reinforcing our understanding of these powerful cosmic sources.





\subsection{Types of AGNs}
Before the unification of the AGNs astornomers named the pusseling objects based on their observational properties. These 
names are still used to this day and are somewhat usfull since their observational properties are important parameters for further study. 
The different classification are important in understanding which objects could have the potential to produce the different oberservables one 
looks for in the night sky. There is a lot of talk around AGNs being possible sources for ultra high energy comsic rays (UHECRs) and neutrinos.
This is yet to be comfirmed, but the theoretical framework for the neceassary particle acceleration is there.




\section{AGN geometry}

\section{ luminosity functions}

*what is a LF
* X-ray LF 
* 

a luminosity function is a function that describes the distribution of objects for a population of celestial objects,
such as galaxies or quasars. It is a powerful tool for understanding the properties and evolution of 
these objects, as well as the larger-scale structure of the universe. 
The function describes how this population varies based on luminosity but also crucially on its comoving volume element. 
 We usually talk about the differential luminosity function given as
\begin{equation}
    \frac{d\Psi(L,z)}{dL} = \frac{d^2N(L,z)}{dLdVc}
\end{equation}

The quantity of interest is now a number density which can be very useful in deriving observed flux of different objects here on earth. 

Several articles express the luminosity function in abase $10$ logarithm and we note the conversion between the two. 

\begin{equation}
    \frac{d\Psi(Lx,z)}{dLog(Lx)} =  \ln (10)  Lx \frac{d\Psi(Lx,z)}{d(Lx)}
\end{equation}

\subsection{X-ray LF}

\begin{table}
\tiny
\begin{tabularx}{\textwidth}{|X|X|X|X|X|X|X|X|X|X|X|}
\hline
Model & A & $L_{star}$ & $\gamma _1$ &  $\gamma _2$  & $v_1$ & $v_2$ & $z_c$ & $L_c$ & $ \alpha$ & $corr$-$fac$ \\
\hline
RG-SLDDE & 8.375e-07 & 2.138e+44 & 2.15 & 1.10 & 4.00 & -1.50 & 1.90 & 3.981e+44 & 0.317 & 1 \\
\hline
AMPLE-Blazar & 1.379e-07 & 1.810e+44 & -0.87 & 2.73 & 3.45 & -0.25 & & & & 400 \\
\hline
AMPLE-FSRQ & 1.750e-08 & 2.420e+44 & -50.00 & 2.49 & 3.67 & -0.30 & & & & 400 \\
\hline
APLE-BLlac & 8.300e-08 & 1.000e+44 & 2.61 & -0.79 & & & & & &400 \\
\hline
\end{tabularx}
\caption{X-ray LF parameters}
\end{table}


One way of calculating the neutrino flux of AGNs is based on their connecting with x-ray radiation. Therefore in some literature, it is of interest to define the x-ray luminosity function for AGNs.

In both \cite{Ajello_2009} and \cite{Ueda_2003} they calculate the luminsity function for different types of AGNs



This is over a particular band of wavelengths


This equation does not have a simple equation to describe all types of light sources, but for different intervals of wavelengths and redshifts (z) one can observe different trends for different light sources. 

In the paper \cite{Jacobsen:2015mga} the author uses the xray luminosity function. The X-ray luminosity function is used to describe the distribution of X-ray luminosities of objects in a specific population, such as galaxies, galaxy clusters, or active galactic nuclei (AGNs). It provides information about the number density of objects at various X-ray luminosity levels within a given volume of the universe.


In \cite{Jacobsen:2015mga} there is a mention of several different models for different populations of AGNs. There she highlights two, \cite{Ajello_2009} and \cite{Ueda_2003}. They are used for different populations.

The present day XLF is presented in \cite{Ajello_2009} and is given by a simple power law. 

\begin{equation}
    \frac{d\Psi(Lx,0)}{dLog(Lx)} = A\ln (10){(\frac{Lx}{Lc})}^{(1-\gamma_2)}
\end{equation}

However there is a break with high enough score count and this break can be better fitted with a double power law 

\end{document}

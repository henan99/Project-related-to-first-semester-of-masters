\section{Introduction}
\subsection{Motivation and background}
Cosmic rays and neutrinos have become an important tool 
for studying the Universe. Their unique properties allow them to probe different regions better than
regular photons and are therefore a great complementary tool in observational astronomy. 
Although the information gain of detecting Cosmic rays and neutrinos are great, they are not without their challenges, 
especially when one moves into higher energies where the fluxes are low.  

The biggest challenge relating to ultra-high-energy cosmic rays (UHECR) and high energy neutrinos is their source.
Both of these particles are thought to be produced in the most energetic events/objects in the Universe, and as such the search for these 
sources are of great interest. One of the candidates is Active Galactic Nuclei (AGN). These are some of the most energetic objects in the Universe, powered by 
accretion of matter onto supermassive black holes. AGN are a fairly new object in astronomy, and one can separate them into several subclasses based on orientation, all with different potential
to produce UHECRs and neutrinos. 


\subsection{Outline}
The main goal of this report is to discuss the evolution of different classes of AGN, and calculate their emissivity of UHECRs and neutrinos based on the x-ray luminosity.
This emissivity will then be compared to the emissivity of UHECRs and neutrinos as seen by ground based 
telescopes here on Earth. To do this one will consider the distribution of the different classes in both Luminosity and redshift using luminosity functions generated by surveys looking in the x-ray band.

One starts in chapter 2 with a brief introduction to distance measures in cosmology. In chapter 3 one will discuss UHECRs and neutrinos, how they are produced, how they interact when propagating through the Universe, how they are detected and importantly the estimate of their respective emissivity.
Chapter 4 will discuss AGN, their structure, and their subclasses. In chapter 5 one introduce Luminosity functions, and the functional forms of the Luminosity functions used in this report along with their parameters.
Then, in chapter 6 one will discuss the results of the Luminosity functions. In chapter 7 one will discuss the results of the emissivity calculations, and compare them to the results of the Luminosity functions. 
Finally, in chapter 8 a conclusion and future outlook is given. 
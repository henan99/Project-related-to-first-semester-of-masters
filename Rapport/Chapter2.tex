\section{The ever-expanding Universe}
To investigate sources very far away from an observer it is important to understand the influence this distance 
has on the desired observables. Therefore, in astrophysics and astronomy in general there are distances created to take into account the effects of an expanding Universe. 
This chapter draws heavily from \cite{hogg2000distance}. 


\subsection{Cosmological parameters}

A reasonable place to start is with the Hubble constant $H_0$. 
This parameter sets the recession speed of a point at proper distance $d$ and the current position via the relation $v = H_0 d$. 
The subscript $0$ refers to the present epoch signifying that $H_0$ is not static but changes with time. 
The precise value of $H_0$ is quite debated, so it's commonly expressed in a parameterised form,
$$
H_0= 100\frac{\rm km}{\rm s}\frac{1}{\rm Mpc} h.
$$
The parameter $h$ is a dimensionless number that according to current knowledge can take the value between $0.5$ to $0.8$ reflecting the range of answers collected from recent work. 

Beyond its basic definition, $H_0$ also allows for the derivation of two significant cosmic scales:

\textbf{Hubble Time ($t_H$) }: Defined as the inverse of 
$H_0$, $t_H$ provides an estimate of the age of the Universe. 
It sets a scale for the time since the Big Bang, assuming the Universe has been expanding at a constant rate. The equation 
$t_H = 1/H_0 \approx 14 \quad\text{Billion years}$ offers a way to approximate this expansion timescale.


\textbf{Hubble Distance ($D_H$) }: This is a measure of the distance. Calculated as 
$D_H = c/H_0 \approx 4.4 \quad \text{Gly}$, where $c$ is the speed of light, 
it represents a critical boundary in observational cosmology. %Beyond this distance, the expansion of the Universe dominates the motion of galaxies, providing a fundamental constraint for cosmological observations and theories.

\subsection{Shape of the Universe}


The shape and expansion of the Universe are central themes in cosmology, but to model such one needs to define the structure of the Universe and its contents. 
In this report and many articles, the Universe is
often explored through the lens of the flat Lambda Cold Dark Matter ($\Lambda$CDM) model. 
This model, widely accepted in contemporary cosmology, provides a framework for understanding the Universe's composition and its expansion dynamics by assuming as the name suggests no curvature and cold dark matter.
In the $\Lambda$CDM model, two key parameters are important: the mass density of the Universe, $\rho_0$, and the cosmological constant, $\Lambda$.
These parameters, which evolve, are a part of defining the metric tensor in general relativity, thereby allowing us to model the curvature of the Universe based on its initial conditions.
These parameters are often expressed as dimensionless variables:

$$
\Omega_m = \frac{8\pi G\rho_0}{3H_0^2}
$$

$$
\Omega_\Lambda = \frac{\Lambda c^2}{3H_0^2}
$$

Here, $\Omega_m$ represents the matter density parameter, encompassing both ordinary (baryonic) matter and dark matter. 
$\Omega_\Lambda$, on the other hand, corresponds to the density parameter associated with the cosmological constant, which is often interpreted as dark energy.




In general, one has a third density parameter $\Omega_k$ which defines the curvature of space-time and the relationship between these parameters is expressed as: 

$$
\Omega_m + \Omega_\Lambda + \Omega_k = 1
$$


In a flat Universe, one has $\Omega_k = 0$ and the Universe is dominated by dark energy and dark matter. The model used in this report and the papers used in the following chapters is the flat $\Lambda$CDM model where the parameters take the values of 
$\Omega_\Lambda = 0.7$ and $\Omega_m = 0.3$. These values align with current observational data.



\subsection{Redshift}
Redshift is defined as the fractional Doppler shift of emitting light. The Doppler effect is a known effect on different observables in the Universe where the relative motion of sources to observers will impact the observable. The redshift is quantified for a light source as 

\begin{equation}
    z = \frac{\nu_e}{\nu_o}-1 = \frac{\lambda_o}{\lambda_e}-1
\end{equation}

Here $o$ refers to the observed quantity and $e$ the emitted. Due to the expansion of the Universe the light emitted from a distant source will be increasingly redshifted the further away it is.
In these scenarios the redshift serves as a distance measure, allowing us to deduce distances to faraway objects.



\subsection{Comoving distance}
\label{sec:comoving_distance}


Comoving distance is an important concept in cosmography, 
acting as a standard unit for various distance measurements in the Universe. 
This distance, often termed the line-of-sight distance for an observer on Earth, 
remains constant even as objects expand with the Hubble flow. 
To calculate the total comoving distance ($D_c$) to an object, 
one integrates the differential comoving distances ($\delta D_c$) along the line of sight, starting from redshift 
$z=0$ to the object. This integration necessitates consideration of the Universe's parametric composition and the $\delta D_c$ is expressed as

\begin{equation}
    \delta D_c = \frac{D_H}{E(z)}dz,
\end{equation}
where the function $E(z)$ is defined as
\begin{equation}
    E(z)  = \sqrt{\Omega_m(z+1)^3 +\Omega_k (1+z)^2 + \Omega_\Lambda  }.
\end{equation}
Here, 
$E(z)$ incorporates the density parameters previously discussed and the redshift 
$z$. It also relates to the Hubble constant observed by a hypothetical observer at redshift $z$, expressed as 
$H(z) = H_0 E(z)$.

One then calculates the comoving distance $D_c$ from 
\begin{equation}
    D_c =D_H \int_0^z\frac{dz}{E(z)}
\end{equation}

In addition to the line of sight, one can define the transverse comoving distance $D_m$. This distance 
relates two points in the night sky at the same redshift separated by an angle $d\theta$. The actual distance
between them $d\theta D_m$ will then vary depending on the curvature of the Universe. This relationship is summarized in the following equation
which accounts for different geometries,

$$
D_m =
\begin{cases}
  D_h\frac{1}{\sqrt{\Omega_k}}sinh(\frac{\sqrt{\Omega_k}D_c}{D_H}) & \text{if } \Omega_k > 0 \\
  D_c& \text{if } \Omega_k = 0 \\
  D_h\frac{1}{\sqrt{|\Omega_k|}}sin(\frac{\sqrt{|\Omega_k|}D_c}{D_H}) & \text{if } \Omega_k < 0
\end{cases}
$$

The different cases correspond to hyperbolic, flat, and spherical geometry respectively. The true nature 
of the Universe is still unknown, but recent observations indicate a flat Universe. 








\subsection{Luminosity distance}
The luminosity distance $D_l$ is defined through the relation between 
the bolometric flux $F$ of a source and its bolometric luminosity $L$. Bolometric flux is the energy received per unit of time per unit area without any obscuration, while bolometric luminosity is the total energy emitted per unit of time.
The luminosity distance is defined as
\begin{equation}    
    D_l = \sqrt{\frac{L}{4\pi F}}
\end{equation}


It is related to the transverse comoving distance via 

\begin{equation}
    D_l = (1+z)D_m.
\end{equation}

If one wants to calculate the spectral 
flux/ differential flux one needs to take into account a correction. This correction comes 
from the fact that one is viewing a redshifted object. The object is emitting in a different band than 
observed. The spectrum of the differential flux $F_\nu$ is related to the spectral luminosity via
\begin{equation}
    F_\nu = (1+z) \frac{L_{(1+z)\nu}}{L_\nu}\frac{L_\nu}{4\pi D_l^2}.
\end{equation}


All these equations listed help include the effects of an expanding Universe when astronomers study distant objects and their properties.
